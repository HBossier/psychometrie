
\OPGAVE
{
\section{Generaliseerbaarheidsstudie}

\begin{enumerate}

\item
Het practicum ontwikkelingspsychologie II bestaat uit een schriftelijk verslag met 10 vragen dat elke student dient in te vullen. Deze worden door twee beoordelaars ge\"{e}valueerd waarbij elke beoordelaar alle 10 de vragen scoort.\\ 
Na toepassing van een variantie-analyse wordt de volgende schatting van de variantiecomponenten bekomen:

\begin{center}
\renewcommand{\arraystretch}{1.2}
\begin{tabular}{|c|c|c|c|c|c|c|c|} \hline
 & $ \hat{\sigma}^2_{s} $ & $ \hat{\sigma}^2_{v} $& $ \hat{\sigma}^2_{b} $ & $ \hat{\sigma}^2_{sv} $ & $ \hat{\sigma}^2_{sb} $ & $ \hat{\sigma}^2_{vb} $ & $ \hat{\sigma}^2_{svb,e} $ \\ \hline
Waarde  & $ 0.397 $ & $ 0.109 $ & $ 0.010 $ & $ 0.314 $ & $ 0.067 $ & $ 0.006 $ & $ 0.224 $ \\ 
\% Var & $ 35 $ & $ 10 $ & $ 1 $ & $ 28 $ & $ 6 $ & $ 1 $ & $ 20 $ \\ \hline
\end{tabular}
\end{center}


\normalsize
De gegevens uit de tabel kunnen gebruikt worden om volgende vragen te beantwoorden.

\begin{enumerate}
\item JUIST of FOUT: "Idealiter is $ \hat{\sigma}^2_{s} $ substantieel groter dan $ \hat{\sigma}^2_{v}$``?
\item Stel, de verantwoordelijke lesgever zou graag in volgende jaren de werklast voor de beoordelaars opsplitsen (situatie $b$). Beoordelaar 1 zou dan enkel vraag 1-5 verbeteren, terwijl beoordelaar 2 vraag 6-10 zou verbeteren. Hoe ziet ons model eruit, gebruik makende van de symbolen $s$, $v$ en $b$?
\item Bereken de generaliseerbaarheidsco\"{e}ffici\"{e}nt (G) voor deze nieuwe situatie ($b$).

\item Stel, de verantwoordelijke lesgever vraagt zich af of wel alle 10 de vragen noodzakelijk zijn. Bepaal met behulp van een $D$-studie de hoeveelheid vragen die minimaal nodig is (gebruik makend van 2 beoordelaars) om een meetnauwkeurigheid van 0.80 te behouden.


\end{enumerate}
\end{enumerate}
}

\OPLOSSING
{
\textbf{Oplossingen}
\begin{enumerate}

\item
\begin{enumerate}
\item JUIST.

\end{enumerate} 
\end{enumerate}
}



