
\OPGAVE
{
\section{Generaliseerbaarheidsstudie}

\begin{enumerate}

\item
Het practicum ontwikkelingspsychologie II bestaat uit een schriftelijk verslag met 10 vragen dat elke student dient in te vullen. Deze worden door twee beoordelaars ge\"{e}valueerd waarbij elke beoordelaar alle 10 de vragen scoort.\\ 
Na toepassing van een variantie-analyse wordt de volgende schatting van de variantiecomponenten bekomen:

\begin{center}
\renewcommand{\arraystretch}{1.2}
\begin{tabular}{|c|c|c|c|c|c|c|c|} \hline
 & $ \hat{\sigma}^2_{s} $ & $ \hat{\sigma}^2_{v} $& $ \hat{\sigma}^2_{b} $ & $ \hat{\sigma}^2_{sv} $ & $ \hat{\sigma}^2_{sb} $ & $ \hat{\sigma}^2_{vb} $ & $ \hat{\sigma}^2_{svb,e} $ \\ \hline
Waarde  & $ 0.397 $ & $ 0.109 $ & $ 0.010 $ & $ 0.314 $ & $ 0.067 $ & $ 0.006 $ & $ 0.224 $ \\ 
\% Var & $ 35 $ & $ 10 $ & $ 1 $ & $ 28 $ & $ 6 $ & $ 1 $ & $ 20 $ \\ \hline
\end{tabular}
\end{center}


\normalsize
De gegevens uit de tabel kunnen gebruikt worden om volgende vragen te beantwoorden.

\begin{enumerate}
\item JUIST of FOUT: ``Idealiter is $ \hat{\sigma}^2_{s} $ substantieel groter dan $ \hat{\sigma}^2_{v}$"?
\item Stel, de verantwoordelijke lesgever zou graag in volgende jaren de werklast voor de beoordelaars opsplitsen (situatie $b$). Beoordelaar 1 zou dan enkel vraag 1-5 verbeteren, terwijl beoordelaar 2 vraag 6-10 zou verbeteren. Hoe ziet ons model eruit, gebruik makende van de symbolen $s$, $v$ en $b$?
\item Bereken de generaliseerbaarheidsco\"{e}ffici\"{e}nt (G) voor deze nieuwe situatie ($b$).

\item Stel, de verantwoordelijke lesgever vraagt zich af of wel alle 10 de vragen noodzakelijk zijn. Bepaal met behulp van een $D$-studie de hoeveelheid vragen die minimaal nodig is (gebruik makend van 2 beoordelaars) om een meetnauwkeurigheid van 0.80 te behouden.


\end{enumerate}
\end{enumerate}
}

\OPLOSSING
{
\textbf{Oplossingen}
\begin{enumerate}

\item
\begin{enumerate}
\item JUIST.
\item $s \times v(b)$

\end{enumerate} 


\end{enumerate}
}


\OPGAVE
{
\section{Generaliseerbaarheidstheorie - extra voorbeeld 2}

\begin{enumerate}

\item
Een verkorte vorm van de TAT (Thematic Apperception Test), bestaande uit 10 kaarten, wordt afgenomen bij 50
jongvolwassen delinquenten ($d$). 
De TAT is zo opgesteld dat de juveniele delinquent bij elk van de 10 kaarten ($k$) een verhaal moet vertellen dat aansluit bij de tekening op de kaart.
De tien verhalen werden op video opgenomen en later beoordeeld.
3 klinische psychologen ($p$) beoordeelden elk de graad van \emph{rejection of authority} op een schaal van 0 tot 100. 
De finale score op de TAT is het gemiddelde van de 30 metingen per subject.
[naar Hoofdstuk 8 oef 3.a uit Crocker and Algina (2008). \emph{Introduction to Classical and Modern Test Theory}.]

\begin{center}
\renewcommand{\arraystretch}{1.2}
\begin{tabular}{|c|c|c|c|c|c|c|c|} \hline
 & $ \hat{\sigma}^2_{s} $ & $ \hat{\sigma}^2_{k} $& $ \hat{\sigma}^2_{p} $ & $ \hat{\sigma}^2_{sk} $ & $ \hat{\sigma}^2_{sp} $ & $ \hat{\sigma}^2_{kp} $ & $ \hat{\sigma}^2_{skp,e} $ \\ \hline
Waarde  & $ 167.64 $ & $ 3.211 $ & $615.8 $ & $ 1.3 $ & $ 84.7 $ & $ 1.3 $ & $ 1.2 $ \\ 
\% Var & 0.1923& 0.004& 0.704& 0.001& 0.097& 0.001& 0.001 \\ \hline
\end{tabular}
\end{center}
% v <- c(167.64,3.211,615.8,1.3,84.7,1.3,1.2)

\normalsize
De gegevens uit de tabel kunnen gebruikt worden om volgende vragen te beantwoorden.

\begin{enumerate}
	\item Schrijf deze opstelling uit.
	\item JUIST of FOUT: ``In deze opstelling kunnen er meer variantie-componenten worden onderscheiden dan wanneer elke $p$ een beperkte (unieke) set van vragen toebedeeld krijgt. Bvb. $p_1$ beoordeelt kaarten 1-4, $p_2$ kaarten 5-8 en tot slot $p_3$ kaarten 9 en 10"?
	\item Bereken de generaliseerbaarheidsco\"{e}ffici\"{e}nt (G) voor dit design.
	\item \emph{tip: schrijf bij deze vraag telkens eerst het opzet uit}\\
	Welke variantie-componenten kunnen niet meer van elkaar onderscheiden worden indien ... 
\begin{itemize}
	\item ... $p_1$ vragen 1-4 beoordeelt, $p_2$ vragen 5-8 en $p_3$ vragen 9 en 10?
	\item ... door tijdsgebrek 25 subjecten kaarten 1-3 aangeboden kregen en 25 andere subjecten kaarten 4-6; bovendien beoordeelt $p_1$ kaarten 1 en 4, $p_2$ kaarten 2 en 5 en $p_3$ kaarten 3 en 6.
\end{itemize}
\item Bereken voor het eerste scenario de $G$-co\"{e}ffici\"{e}nt en voor het tweede scenario de index of dependability $\phi$. Waar ligt het verschil tussen beide maten?
\end{enumerate}
\end{enumerate}
}

\OPLOSSING
{
\textbf{Oplossingen}
\begin{enumerate}
\item JUIST.
\item $s \times k \times p$
\item 
\begin{itemize}
	\item Het opzet wordt hier: $s\times p\left(k\right)$ bijgevolg kunnen de variantiecomponenten $\sigma_p$ en $\sigma_{kp}$ niet onderscheiden worden van elkaar (wordt $\sigma_{pk,p}$) en ook $\sigma_{sp}$ kan niet van de residuele variantie-term worden onderscheiden (wordt $\sigma_{sp, spk,e}$).
	\item Het opzet wordt hier: $k\left(p\times s\right)$ bijgevolg kunnen de variantiecomponenten $\sigma_k$ en $\sigma_{kp}$, $\sigma_{ks}$ niet onderscheiden worden van  van de residuele variantie-term worden onderscheiden (wordt $\sigma_{k, kp, ks, spk, e}$).  
\end{itemize}
\item NOG UIT TE REKENEN.
\end{enumerate} 
}

