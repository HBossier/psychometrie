
\OPGAVE
{
\section{Generaliseerbaarheidsstudie}

\begin{enumerate}

\item
Het practicum ontwikkelingspsychologie II bestaat uit een schriftelijk verslag met 10 vragen dat elke student dient in te vullen. Deze worden door twee beoordelaars ge\"{e}valueerd waarbij elke beoordelaar alle 10 de vragen scoort.\\ 
Na toepassing van een variantie-analyse wordt de volgende schatting van de variantiecomponenten bekomen:

\begin{center}
\renewcommand{\arraystretch}{1.2}
\begin{tabular}{|c|c|c|c|c|c|c|c|} \hline
 & $ \hat{\sigma}^2_{s} $ & $ \hat{\sigma}^2_{v} $& $ \hat{\sigma}^2_{b} $ & $ \hat{\sigma}^2_{sv} $ & $ \hat{\sigma}^2_{sb} $ & $ \hat{\sigma}^2_{vb} $ & $ \hat{\sigma}^2_{svb,e} $ \\ \hline
Waarde  & $ 0.397 $ & $ 0.109 $ & $ 0.010 $ & $ 0.314 $ & $ 0.067 $ & $ 0.006 $ & $ 0.224 $ \\ 
\% Var & $ 35 $ & $ 10 $ & $ 1 $ & $ 28 $ & $ 6 $ & $ 1 $ & $ 20 $ \\ \hline
\end{tabular}
\end{center}


\normalsize
De gegevens uit de tabel kunnen gebruikt worden om volgende vragen te beantwoorden.

\begin{enumerate}
\item JUIST/FOUT: idealiter is $ \hat{\sigma}^2_{s} $ substantieel groter dan $ \hat{\sigma}^2_{v}$.
\item Stel, de verantwoordelijke lesgever zou graag in volgende jaren de werklast voor de beoordelaars opsplitsen (situatie $b$). Beoordelaar 1 zou dan enkel vraag 1-5 verbeteren, terwijl beoordelaar 2 vraag 6-10 zou verbeteren. Hoe ziet ons model er nu uit, gebruik makende van de symbolen $s$, $v$ en $b$?
\item Bereken de generaliseerbaarheidsco\"{e}ffici\"{e}nt (G) voor deze nieuwe situatie ($b$).

\item Volgens een geplande $D$-studie, 

\item Een 'Totale leesscore' kan gevormd worden door de scores van 'Woord vaardigheden' op te tellen bij de scores van 'Begrijpend lezen'. Wat is de variantie van deze nieuwe variabele 'Totale leesscore'?

%\item Wat is de betrouwbaarheid van 'Totale leesscore'?

%\item Een cognitieve psycholoog vormt een nieuwe variabele door de scores voor 'Woordenschat', 'Taal' en 'Sociale studies' samen te voegen. Wat is de correlatie tussen deze nieuwe variabele en 'Totale leesscore'?  (Tip: Stel $X_1 = Y_2 + Y_3$ en $X_2 = Y_1 + Y_4 + Y_7$)

\item Als een persoon een score van 22 haalt op de subtest 'Wetenschappen', wat zijn dan de intervalgrenzen van het $68\%$-betrouwbaarheidsinterval van zijn/haar ware score ($z_{.84} = .994$).

\item Als een persoon een score haalt van 24 op de sutbest 'Wiskunde begrippen', wat zijn dan de intervalgrenzen van het $80\%$-betrouwbaarheidsinterval van zijn/haar ware score ($z_{.90}=1.28$)? 

\end{enumerate}
\end{enumerate}
}

\OPLOSSING
{
\textbf{Oplossingen}
\begin{enumerate}

\item
\begin{enumerate}
\item Deze correlatie kunnen we rechtstreeks uit de tabel halen: .55.
\item In de tabel krijgen we de correlaties tussen en de standaardafwijkingen van de verschillende variabelen. Op basis van de formule voor het berekenen van een correlatie,
\begin{displaymath}
\rho_{Y_iY_j}=\frac{cov_{Y_iY_j}}{\sigma_{Y_i}\sigma_{Y_j}}
\end{displaymath}
kunnen we dus de covariantie tussen twee variabelen $Y_i$ en $Y_j$ berekenen:
\begin{displaymath}
cov_{Y_iY_j}=\rho_{Y_iY_j} \times \sigma_{Y_i}\sigma_{Y_j}
\end{displaymath}
De covariantie tussen 'Woord Vaardigheden' en 'Sociale studies' is bijgevolg gelijk aan:
\begin{align*}
cov_{Y_2Y_7}=& \rho_{Y_2Y_7} \times \sigma_{Y_2}\sigma_{Y_7}\\
            =& .55 \times 8.79 \times 4.33\\
            =& 20.93
\end{align*}

\item Uit de cursus statistiek 1 weten we dat de variantie van een som van twee kansvariabelen gelijk is aan de som van de varianties van de afzonderlijke variabelen plus 2 maal de covariantie van beide.
\begin{displaymath}
Var(X) = Var(Y_i) + Var(Y_j) + 2Cov(Y_i,Y_j)
\end{displaymath}
Wanneer we deze formule veralgemenen naar meerdere toevalsvariabelen krijgen we:
\begin{displaymath}
Var(X) = Var(\Sigma_iY_i) = \Sigma_iVar(Y_i) + 2\Sigma_{i>j}Cov(Y_i,Y_j)
\end{displaymath}






\end{enumerate} 
\end{enumerate}
}



